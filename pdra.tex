\documentclass[11pt,a4paper,uplatex,twoside,dvipdfmx]{ujarticle} 	% for uplatex
%\documentclass[11pt,a4paper,twoside,dvipdfmx]{jarticle}		% platex
%==== 科研費LaTeX =============================================
%	2021(R03)年度 海外特別研究員
%============================================================
% 2009-03-07: Taku Yamanaka (Osaka Univ.)
%			Copied over from PD.
% 2011-03-20: Taku Yamanaka
%			Made a 2012 version.
% 2012-02-25: Taku: Made 2013 version.
% 2013-03-15: Taku: Made 2014 version.
% 2014-03-02: Taku: Made 2015 version.
% 2015-02-22: Taku: Made 2016 version.
% 2016-03-16: Taku: Made 2017 version.
%============================================================
%=======================================
% form00_header.tex
%	General header for kakenhiLaTeX,  Moved over from form00_2010_header.tex.
%	2009-09-06 Taku Yamanaka (Osaka Univ.)
%==== General Version History ======================================
% 2006-05-30 Taku Yamanaka (Physics Dept., Osaka Univ.)
% 2006-06-02 V1.0
% 2006-06-14 V1.1 Use automatic calculation for cost tables.
% 2006-06-18 V1.2 Split user's contents and the format.
% 2006-06-20 V1.3 Reorganized user and format files
% 2006-06-25 V1.4 Readjusted all the table column widths with p{...}.
%				With \KLTabR and \KLTabRNum, now the items can be right-justified
%				in the cell defined by p{...}.
% 2006-06-26 V1.5 Use \newlength and \setlength, instead of \newcommand, to define positions.
% 2006-08-19 V1.6 Remade it for 2007 JFY version.
% 2006-09-05 V1.7 Added font declarations suggested by Hoshino@Meisei Univ.
% 2006-09-06 V1.8 Introduced usePDFform flag to switch the form file format.
% 2006-09-09 V1.9 Changed p.7, to allow different heights between years. (Thanks to Ytow.)
% 2006-09-11 V2.0 Added an option to show budget summary.
% 2006-09-13 V2.1 Added an option to show the group.
% 2006-09-14 V2.1.1 Cleaned up Kenkyush Chosho.
% 2006-09-21 V2.2 Generated under a new automatic development system.

% 2007-03-24 V3.0 Switched to a method using "picture" environment.

% 2007-08-14 V3.1 Switched to kakenhi3.sty.
% 2007-09-17 V3.2 Added \KLMaxYearCount
% 2008-03-08 V3.3 Remade it for 2009 JFY version\
% 2008-09-08 V3.4 Added \KLXf ... \KLXh.
% 2011-10-20 V5.0 Use kakenhi5.sty, to utilize array package in tabular environment.
% 2012-08-14 v5.1 Moved preamble and kakenhi5 into the current directory, instead of the parent directory.
% 2012-11-10 v6.0 Switched to kakenhi6.sty.
% 2015-08-26 v6.1 Added KLFirstPageIsLongPage flag.
% 2018-02-12 Taku: Commented out \DeclareFontShape ...
%=======================================
%============================================================
% preamble.tex
%
% Dummy section and subsection commands.
% With these, some editors (such as TeXShop, etc.) can jump to the (sub)sections.
\newcommand{\dummy}{dummy}% 
\renewcommand{\section}[1]{\renewcommand{\dummy}{#1}}
\renewcommand{\subsection}[1]{\renewcommand{\dummy}{#1}}

% Flag for switching form file format.......
\usepackage{ifthen}
\newboolean{usePDFform}
\newboolean{BudgetSummary}

\usepackage{forms/kakenhi6}

\pagestyle{empty}

% ===== Parameters for LaTeX =========================

% ===== Font declarations  ======================================
%%\DeclareFontShape{JT1}{mc}{m}{it}{<->ssub * mc/m/n}{}
%%\DeclareFontShape{JY1}{mc}{m}{it}{<->ssub * mc/m/n}{}

% ===== Parameters for KL (Kakenhi LaTeX) ========================
% general purpose temporary variables	-2007
\newcommand{\KLX}{}
\newcommand{\KLXa}{}
\newcommand{\KLXb}{}
\newcommand{\KLXc}{}
\newcommand{\KLXd}{}
\newcommand{\KLXe}{}
\newcommand{\KLXf}{}
\newcommand{\KLXg}{}
\newcommand{\KLXh}{}
\newcommand{\KLY}{}
\newcommand{\KLYa}{}
\newcommand{\KLYb}{}
\newcommand{\KLYc}{}
\newcommand{\KLYd}{}
\newcommand{\KLYe}{}
\newcommand{\KLYf}{}
\newcommand{\KLXR}{}
\newlength{\KLCella}
\newlength{\KLCellb}
\newlength{\KLCellc}
\newlength{\KLCelld}
\newlength{\KLCelle}
\newlength{\KLCellf}
\newlength{\KLCellg}
\newlength{\KLCellh}

% sub-page
\newlength{\KLSubPageX}
\newlength{\KLSubPageY}
\newlength{\KLspx}
\newlength{\KLspy}
\newcommand{\KLSubPageXmm}{}	% for \input(x,y){....} which uses a unit (mm)
\newcommand{\KLSubPageYmm}{}	% for \input(x,y){....} which uses a unit (mm)

% margins for parbox inside frames; in units of points
\newcounter{KLParboxSideMargin}
\newcounter{KLParboxTopMargin}
\newcounter{KLParboxBottomMargin}

% ===== standard counters ======================================
\newcounter{KLSubPageNo}	% sub-page counter
\newcounter{KLPageOffset}		% to generate sub-page number
\newcounter{KLMaxYearCount}	% # of years for the proposal

% ===== standard flags ============================
\newboolean{KLFirstPageIsLongPage}

% ===== initializations ============
\KLInitTypesettingPageSelection
\newcommand{\KLCLLang}{}	% language-dependent left-justification in tabular



% user01_header
%=== 様式のファイルの形式の指定 =================
%   PDFではなく、eps の様式を読み込む場合は、次の行の頭に「%」をつけてください。
\setboolean{usePDFform}{true}
%===================================
%==========================================================
% form01_header.tex
%	2014-03-02: Taku Yamanaka (Osaka Univ.)
%		This is called after usePDFform is set.
%		Originally, this part was in form07_header.tex, but then
%		\usepackage{color} that is called before it was not effective.
%		[dvipdfmx] is not used for eps forms, because it makes the forms
%		slightly larger than pdf forms.
%		
%==========================================================
% ===== File format for forms ===========================
\ifthenelse{\boolean{usePDFform}}{
	\newcommand{\KLFormFormat}{pdf}	\usepackage[dvipdfmx]{graphicx}
}{	\newcommand{\KLFormFormat}{eps}	\usepackage{graphicx}
}

%----------------------------------------------------------------------------


% user02_header
%=== 予算の表の印刷 =====================
% 予算の集計の表を出すためには、次の行の頭の%を消してください。
%\setboolean{BudgetSummary}{true}
%=================================

%=== For English, uncomment the next line to left-justify inside table columns.
%\renewcommand{\KLCLLang}{\KLCL}

% === 一部のページだけタイプセット ==============
% New in 2009 fall version!
% 選んだページだけタイプセットするには、次の例の頭の%を消し、並べてください。
% 複数のページを選ぶこともできます。
% 提出前には、必ず全てコメントアウト(頭に%をつける)してください。
%ーーーーーーーーーーーーーーーーーーーーーーーーーーーーーーーーー
%\KLTypesetPage{1}			% p.1 (or p.1を含む連続したページ),
%\KLTypesetPage{3}			% p.3 (or p.3を含む連続したページ),
%\KLTypesetPagesInRange{5}{6}	% p.5 ~ p.6,
%\KLTypesetPagesInRange{8}{10}	% and p.8 ~ p.10
%=================================

% ===== my favorite packages ====================================
% ここに、自分の使いたいパッケージを宣言して下さい。
\usepackage{wrapfig}
% \usepackage{amssymb}
%\usepackage{mb}
% \usepackage{color} % でも科研費の書類はグレースケールで印刷されます
%\DeclareGraphicsRule{.tif}{png}{.png}{`convert #1 `dirname #1`/`basename #1 .tif`.png}

\usepackage[multi,deluxe,bold,expert]{otf}
\usepackage{comment}
\usepackage[framemethod=tikz]{mdframed}
%==========================================================

\newcommand{\KLShouKeiLine}[1]{\cline{#1}}
%もし、小計の上の線を取れと事務に言われたら、
%「そのようなことは、記入要項に書かれていないし、学振はそのようなことは気にしていない。」と
% 突っぱねる。
% それでもなお消せと理不尽なことを言われたら、次の行の 最初の「%」を消す。	
%\renewcommand{\KLShouKeiLine}[1]{}

\newcommand{\KLBudgetTableFontSize}{small}	% 予算の表のフォントの大きさ: small, footnotesize
\newcommand{\KLFundsTableFontSize}{small}	%応募中、受入れ予定の研究費のフォントの大きさ:normalsize, small, footnotesize

% ===== my personal definitions ==================================
% ここに、自分のよく使う記号などを定義して下さい。
\newcommand{\klpionn}{K_L \to \pi^0 \nu \overline{\nu}}
\newcommand{\kppipnn}{K^+ \to \pi^+ \nu \overline{\nu}}


\renewcommand{\emph}[1]{{\sffamily\gtfamily\bfseries #1}}
\newcommand{\subject}[1]{\noindent{\sffamily\gtfamily\bfseries #1}~~}
\newcommand{\subsubject}[1]{\noindent \underline{#1}~~}
%\newcommand{\Red}[1]{\textcolor{red}{\sffamily\gtfamily\bfseries #1}}
\renewcommand{\bf}{\bfseries\sffamily\gtfamily}

% hook3: after including packages ===================
 % for future maintenance
% ===== Global definitions for the PD form ======================
% 基本情報
%
%------ 研究課題名  -------------------------------------------
\newcommand{\研究課題名}{\mgfamily ストカスティックインフレーション, 原始ブラックホール, および重力波観測}

%----- 研究機関名と研究代表者の氏名-----------------------
\newcommand{\研究機関名}{\mgfamily 名古屋大学}
\newcommand{\申請者氏名}{\mgfamily 多田 祐一郎}
\newcommand{\研究代表者氏名}{\申請者氏名}

%---- 研究期間の最終年度 ----------------
\newcommand{\研究期間の最終元号年度}{34}	%平成で、半角数字のみ
%=========================================================
% ===== Global year-dependent definitions for the Kakenhi form ===========
% 基本情報
\newcommand{\研究開始年度}{2021}
\newcommand{\研究開始元号年度}{03}	%令和

\newcommand{\一年目西暦}{2021}
\newcommand{\二年目西暦}{2022}
\newcommand{\三年目西暦}{2023}
\newcommand{\四年目西暦}{2024}
\newcommand{\五年目西暦}{2025}
\newcommand{\六年目西暦}{2026}

\newcommand{\一年目}{3}
\newcommand{\二年目}{4}
\newcommand{\三年目}{5}
\newcommand{\四年目}{6}
\newcommand{\五年目}{7}
\newcommand{\六年目}{8}

\newcommand{\一年目J}{3}
\newcommand{\二年目J}{4}
\newcommand{\三年目J}{5}
\newcommand{\四年目J}{6}
\newcommand{\五年目J}{7}
\newcommand{\六年目J}{8}


	% <<<
%==========================================================
% form03_header.tex
%	2009-03-04: Taku Yamanaka (Osaka Univ.)
%==========================================================
\usepackage{calc}
\usepackage{watermark}
\usepackage{longtable}
\usepackage{geometry}                % See geometry.pdf to learn the layout options. There are lots.
\usepackage{udline}
\usepackage{array}

\geometry{noheadfoot,scale=1}  %scale=1 resets margins to 0
\setlength{\unitlength}{1pt}

% define variables for positions ==========================
% picture environment location, in  units of points
\newcommand{\KLOddPictureX}{}
\newcommand{\KLEvenPictureX}{}
\newcommand{\KLPictureY}{}
\newcommand{\KLOddPictureInWaterMarkX}{}
\newcommand{\KLEvenPictureInWaterMarkX}{}
\newcommand{\KLPictureInWaterMarkY}{}

\newlength{\KLoddsidemargin}
\newlength{\KLevensidemargin}
\newlength{\KLtopmargin}

\newcounter{KLCOddPictureInWaterMarkX}
\newcounter{KLCEvenPictureInWaterMarkX}
\newcounter{KLCPictureInWaterMarkY}
\newcounter{KLCOddPictureX}
\newcounter{KLCEvenPictureX}
\newcounter{KLCPictureY}

%------------------------------------------------------------

\newcommand{\KLLeftEdge}{}
\newcommand{\KLRightEdge}{}

% standard margins for text in frames
\setcounter{KLParboxSideMargin}{7}
\setcounter{KLParboxTopMargin}{12}
\setcounter{KLParboxBottomMargin}{5}

%-----------------------------------------------------------
\newcommand{\KLTwoHLines}{\hline\hline}



%=================================================================
% form05_pdra_header.tex
%	2010-03-07: Taku Yamanaka (Osaka Univ.)
%			Copied from PD.
%=================================================================

% ===== Global definitions for the Kakenhi form ======================
% 基本情報
\newcommand{\研究種目}{海外特別研究員}
\newcommand{\研究種目後半}{}
\ifthenelse{\isundefined{\研究種別}}{
	\newcommand{\研究種別}{}
}{}%
\newcommand{\KLMainFile}{pdra.tex}
\newcommand{\KLForms}{pdra_forms}
\newcommand{\KLYoshiki}{pdra}

% 奇数ページの下に記入される情報
\newcommand{\klbyYup}{}
\newcommand{\klbyYdown}{}
\newcommand{\klbyKikanXleft}{}
\newcommand{\klbyKikanXright}{}
\newcommand{\klbyNameXleft}{}
\newcommand{\klbyNameXright}{}

\newcommand{\KLBottomInfo}[6]{%
	\ifthenelse{\equal{#1}{}}{%
		\renewcommand{\klbyYup}{54}
		\renewcommand{\klbyYdown}{43}
	}{%
		\renewcommand{\klbyYup}{#1}
		\renewcommand{\klbyYdown}{#2}
	}
	
	\ifthenelse{\equal{#3}{}}{%
		\renewcommand{\klbyKikanXleft}{132}
		\renewcommand{\klbyKikanXright}{349}
		\renewcommand{\klbyNameXleft}{415}
		\renewcommand{\klbyNameXright}{542}
	}{%
		\renewcommand{\klbyKikanXleft}{#3}
		\renewcommand{\klbyKikanXright}{#4}
		\renewcommand{\klbyNameXleft}{#5}
		\renewcommand{\klbyNameXright}{#6}
	}
%	\KLTextBox{\klbyKikanXleft}{\klbyYup}{\klbyKikanXright}{\klbyYdown}{}{\研究機関名}%
	\KLTextBox{\klbyNameXleft}{\klbyYup}{\klbyNameXright}{\klbyYdown}{}{\申請者氏名}%
}

%==========================================================
% frame edge positions of multi-page-block
\newcommand{\KLOddMultiPageLeftEdge}{51}
\newcommand{\KLOddMultiPageRightEdge}{544}
\newcommand{\KLEvenMultiPageLeftEdge}{51}
\newcommand{\KLEvenMultiPageRightEdge}{544}

% vertical limits in the first multi-page-block
\newcommand{\KLMultiPageTopEdge}{785}		%lowest top position (except for the 1st page)
\newcommand{\KLMultiPageBottomEdge}{71}	%highest bottom position (except for the last page)

% Modify the edges for single page frames if necessary
\newcommand{\KLOddLeftEdge}{\KLOddMultiPageLeftEdge}
\newcommand{\KLOddRightEdge}{\KLOddMultiPageRightEdge}
\newcommand{\KLEvenLeftEdge}{\KLEvenMultiPageLeftEdge}
\newcommand{\KLEvenRightEdge}{\KLEvenMultiPageRightEdge}

%

%==========================================================
% form07_header.tex
%	2009-03-04: Taku Yamanaka (Osaka Univ.)
%	2014-03-02: Taku: Moved graphics part to form01_header.tex.
%	2015-08-26: Taku: Added a test for \KLFirstPageIsLongPage.
%==========================================================
% Remember Standard Positions that were set in form05_xxxx_header.tex
\let \KLStandardOddMultiPageLeftEdge = \KLOddMultiPageLeftEdge
\let \KLStandardOddMultiPageRightEdge = \KLOddMultiPageRightEdge
\let \KLStandardEvenMultiPageLeftEdge = \KLEvenMultiPageLeftEdge
\let \KLStandardEvenMultiPageRightEdge = \KLEvenMultiPageRightEdge

\let \KLStandardMultiPageTopEdge = \KLMultiPageTopEdge
\let \KLStandardMultiPageBottomEdge = \KLMultiPageBottomEdge

\let \KLStandardOddLeftEdge = \KLOddLeftEdge
\let \KLStandardOddRightEdge = \KLOddRightEdge
\let \KLStandardEvenLeftEdge = \KLEvenLeftEdge
\let \KLStandardEvenRightEdge = \KLEvenRightEdge

%------ This should be set before \begin{document} ------
\KLStandardLengths
\KLStandardPositions

\ifthenelse{\boolean{KLFirstPageIsLongPage}}{%
	\setlength{\textheight}{10000pt}%
}{%
}
%----------------------------------------------------------------------------


%============================================================
%endPrelude

\begin{document}
\mgfamily\sffamily
% hook5 : right after \begin{document} ==============
\newcommand{\応募者の研究遂行能力}{}		% patch 2020-04-05
 % for future maintenance
%============================================================
%     User Inputs
%============================================================

%form: pdra_form_04-05.tex ; user: pdra_04-05_preparation_etc.tex
%========== 海外特別研究員 =========
%===== p. 04-05 現在までの研究状況 =============
\section{現在までの研究状況}
%watermark: w03_past_pdra
\newcommand{\現在までの研究状況}{%
%begin  現在までの研究状況===================
	
	%\begin{mdframed}[roundcorner=0.5zw,
	%%skipabove=1zw,skipbelow=1zw,
	%innertopmargin=0.8zw,innerbottommargin=0.8zw,
	%%innerleftmargin=0.8zw,innerrightmargin=0.8zw,
	%%rightmargin=5000pt,leftmargin=50pt,
	%linecolor=black!50,linewidth=0.2zw,
	%backgroundcolor=black!10]
	%{\bfseries\gtfamily\sffamily\large 1. 研究の学術的背景および核心となる「問い」}
	%\end{mdframed}

	%今までは、地球上で最大の生物、シロナガスクジラの卵の研究を進めようとしてきた。
	%クジラの卵の場合は、高い水圧に耐える必要があるため、堅固の構造となっているはずであり、
	%これが解明されれば、将来、深海潜水艇への応用も効く。
	%しかし、シロナガスクジラの生息範囲が広い、海に潜っている時間が長い、
	%生息数も減っている、などの原因により、
	%卵を見つけることができなかった。
	
	%そこで、\underline{地球で}最大の動物から、\underline{地上で}最大の動物に研究対象を変更する。

	%ぞうの卵はおいしいぞう。
ぞうの卵はおいしいぞう。
ぞうの卵はおいしいぞう。
ぞうの卵はおいしいぞう。
ぞうの卵はおいしいぞう。
ぞうの卵はおいしいぞう。
ぞうの卵はおいしいぞう。
ぞうの卵はおいしいぞう。
ぞうの卵はおいしいぞう。
ぞうの卵はおいしいぞう。
ぞうの卵はおいしいぞう。
ぞうの卵はおいしいぞう。
ぞうの卵はおいしいぞう。
ぞうの卵はおいしいぞう。
ぞうの卵はおいしいぞう。
ぞうの卵はおいしいぞう。
ぞうの卵はおいしいぞう。
ぞうの卵はおいしいぞう。
ぞうの卵はおいしいぞう。
ぞうの卵はおいしいぞう。
ぞうの卵はおいしいぞう。
ぞうの卵はおいしいぞう。
ぞうの卵はおいしいぞう。
ぞうの卵はおいしいぞう。
ぞうの卵はおいしいぞう。
ぞうの卵はおいしいぞう。
ぞうの卵はおいしいぞう。
ぞうの卵はおいしいぞう。
ぞうの卵はおいしいぞう。
ぞうの卵はおいしいぞう。
ぞうの卵はおいしいぞう。
ぞうの卵はおいしいぞう。
ぞうの卵はおいしいぞう。
ぞうの卵はおいしいぞう。
ぞうの卵はおいしいぞう。
ぞうの卵はおいしいぞう。
ぞうの卵はおいしいぞう。
ぞうの卵はおいしいぞう。
ぞうの卵はおいしいぞう。
ぞうの卵はおいしいぞう。
ぞうの卵はおいしいぞう。
ぞうの卵はおいしいぞう。
ぞうの卵はおいしいぞう。
ぞうの卵はおいしいぞう。
ぞうの卵はおいしいぞう。
ぞうの卵はおいしいぞう。
ぞうの卵はおいしいぞう。
ぞうの卵はおいしいぞう。
ぞうの卵はおいしいぞう。
ぞうの卵はおいしいぞう。
ぞうの卵はおいしいぞう。
ぞうの卵はおいしいぞう。
ぞうの卵はおいしいぞう。
ぞうの卵はおいしいぞう。
ぞうの卵はおいしいぞう。
ぞうの卵はおいしいぞう。
ぞうの卵はおいしいぞう。
ぞうの卵はおいしいぞう。
ぞうの卵はおいしいぞう。
ぞうの卵はおいしいぞう。
ぞうの卵はおいしいぞう。
ぞうの卵はおいしいぞう。
ぞうの卵はおいしいぞう。
ぞうの卵はおいしいぞう。
ぞうの卵はおいしいぞう。
ぞうの卵はおいしいぞう。
ぞうの卵はおいしいぞう。
ぞうの卵はおいしいぞう。
ぞうの卵はおいしいぞう。
ぞうの卵はおいしいぞう。
ぞうの卵はおいしいぞう。
ぞうの卵はおいしいぞう。
ぞうの卵はおいしいぞう。
ぞうの卵はおいしいぞう。
ぞうの卵はおいしいぞう。
ぞうの卵はおいしいぞう。
ぞうの卵はおいしいぞう。
ぞうの卵はおいしいぞう。
ぞうの卵はおいしいぞう。
ぞうの卵はおいしいぞう。
ぞうの卵はおいしいぞう。
ぞうの卵はおいしいぞう。
ぞうの卵はおいしいぞう。
ぞうの卵はおいしいぞう。
ぞうの卵はおいしいぞう。
ぞうの卵はおいしいぞう。
ぞうの卵はおいしいぞう。
ぞうの卵はおいしいぞう。
ぞうの卵はおいしいぞう。
ぞうの卵はおいしいぞう。
ぞうの卵はおいしいぞう。
ぞうの卵はおいしいぞう。
ぞうの卵はおいしいぞう。
ぞうの卵はおいしいぞう。
ぞうの卵はおいしいぞう。
ぞうの卵はおいしいぞう。
ぞうの卵はおいしいぞう。
ぞうの卵はおいしいぞう。
ぞうの卵はおいしいぞう。
ぞうの卵はおいしいぞう。
ぞうの卵はおいしいぞう。
ぞうの卵はおいしいぞう。
ぞうの卵はおいしいぞう。
ぞうの卵はおいしいぞう。
ぞうの卵はおいしいぞう。
ぞうの卵はおいしいぞう。
ぞうの卵はおいしいぞう。
ぞうの卵はおいしいぞう。
ぞうの卵はおいしいぞう。
ぞうの卵はおいしいぞう。
ぞうの卵はおいしいぞう。
ぞうの卵はおいしいぞう。
ぞうの卵はおいしいぞう。
ぞうの卵はおいしいぞう。
ぞうの卵はおいしいぞう。
ぞうの卵はおいしいぞう。
ぞうの卵はおいしいぞう。
ぞうの卵はおいしいぞう。
ぞうの卵はおいしいぞう。
ぞうの卵はおいしいぞう。
ぞうの卵はおいしいぞう。
ぞうの卵はおいしいぞう。
ぞうの卵はおいしいぞう。
ぞうの卵はおいしいぞう。
ぞうの卵はおいしいぞう。
ぞうの卵はおいしいぞう。
ぞうの卵はおいしいぞう。
ぞうの卵はおいしいぞう。
ぞうの卵はおいしいぞう。
ぞうの卵はおいしいぞう。
ぞうの卵はおいしいぞう。
ぞうの卵はおいしいぞう。
ぞうの卵はおいしいぞう。
ぞうの卵はおいしいぞう。
ぞうの卵はおいしいぞう。
ぞうの卵はおいしいぞう。
ぞうの卵はおいしいぞう。
ぞうの卵はおいしいぞう。
ぞうの卵はおいしいぞう。
ぞうの卵はおいしいぞう。
ぞうの卵はおいしいぞう。
ぞうの卵はおいしいぞう。
ぞうの卵はおいしいぞう。
ぞうの卵はおいしいぞう。
ぞうの卵はおいしいぞう。
ぞうの卵はおいしいぞう。
ぞうの卵はおいしいぞう。
ぞうの卵はおいしいぞう。
ぞうの卵はおいしいぞう。
ぞうの卵はおいしいぞう。
ぞうの卵はおいしいぞう。
ぞうの卵はおいしいぞう。
ぞうの卵はおいしいぞう。
ぞうの卵はおいしいぞう。
  % << only for demonstration. Please delete it or comment it out.	
%end  現在までの研究状況 ====================
}

%form: pdra_form_06-07.tex ; user: pdra_06-07_purpose.tex
%========== 海外特別研究員 =========
%===== p. 06-07 これからの研究計画 =============
\section{これからの研究計画}
%watermark: w02_purpose_pdra
\subsection{研究目的・内容}
\subsection{研究の特色・独創的な点}
\newcommand{\研究の特色と独創的な点}{%
%begin  研究の特色と独創的な点===================
	%象の卵の特色と独創的な点は...
%end  研究の特色と独創的な点 ====================
}

\subsection{研究目的}
\newcommand{\研究目的}{%
%begin  研究目的===================
	%象の卵の研究の目的は...
	
	%\vspace{1cm}
	%\begin{thebibliography}{99}
		%\bibitem{teramura} 寺村輝夫、「ぼくは王様 - ぞうのたまごのたまごやき」.
	%\end{thebibliography}
%end  研究目的 ====================
}

%====================================
%form: pdra_form_08.tex ; user: pdra_08_why_abroad.tex
%========== 海外特別研究員 =========
%===== p. 08 外国で研究する理由など =============
\section{外国で研究する理由など}
\section{外国で研究する事の意義}
\newcommand{\外国で研究する事の意義}{%
%begin  外国で研究する事の意義 (figureやtable使用可)===================
	%私は今まで、象の卵の可能性について主に文献を漁って研究をしてきた。
	%そうした長年の研究の末分かったことの一つは、日本に現在、自然界に生息
	%する象はいないということである。
	%最も最近生息した象はケナガマンモスのようであるが、
	%祖父が子供の頃には既に絶滅していたそうである。
	%マンモスの氷漬けの個体は北海道で見つかったが、卵は見つかっていない。
	%また最近では2005年に愛知県のある会場で氷漬けの個体が見つかったが、
	%これは実は密かにロシアから持ち込まれたものであり、国産象ではない。
	
	%こうした経験から、象の卵を日本で探していても見つからないということを
	%強く実感し、海外で研究する決心をした次第である。
	%特に、象の卵を探す夢を子供の頃に私に与えてくれた
	%Dr. Seussにぜひとも指導を仰ぎたく、師の元に行って研究を行う。
%end  外国で研究する事の意義 (figureやtable使用可) ====================
}

%form: pdra_form_09.tex ; user: pdra_09_need_rights.tex
%========== 海外特別研究員 =========
%===== p. 09 人権、法令など =============
\section{人権、法令など}
\subsection{人権の保護及び法令の遵守}
\newcommand{\人権の保護及び法令等の遵守への対応}{%
%begin  人権の保護及び法令等の遵守への対応 ===================
	本研究は該当しない.
	%象の卵のES細胞の培養、象のクローンの生成などは行わない。
	%象個体を現地から持ち出すことはないので、ワシントン条約ならびに
        %生物多様性条約に抵触しない。また、組換え実験は行なわないので、
        %カルタヘナ議定書にも抵触しない。
%end  人権の保護及び法令等の遵守への対応 ====================
}

%form: pdra_form_10-11.tex ; user: pdra_10-11_publications.tex
%========== 海外特別研究員 =========
%===== p. 10-11 研究業績 =============
\section{研究業績}
%watermark: w14_pub_pdra
% 2008-03-08 Taku
% 2009-03-04 K.S.
% 2010-03-08 Taku: copied from PD
\renewcommand{\応募者の研究遂行能力}{%
%begin  研究遂行能力 ===================
	\vspace{10\baselineskip}
%end  研究遂行能力 ====================
}

\subsection{学術雑誌(紀要・論文集等も含む)に発表した論文及び著書}
\newcommand{\学術雑誌等に発表した論文または著書}{%
%begin  学術雑誌等に発表した論文または著書===================
	\vspace{10\baselineskip}
	
	%\begin{enumerate}
		%\item[](査読有り)%===========================
		%\item \underline{H. Yukawa}$^1$, J. Kara$^2$,
				%``Theory of Elephant Eggs'', 
				%Phys.\ Rev.\ Lett. {\bf 800}, 800-804 (2005). 
				%\label{pub:theoegg}
				
		%\item F.~Ehrlich, \underline{H. Yukawa}$^1$,
				%``You can't Lay an Egg If You're an Elephant'', 
				%JofUR\\
				 %({\tt www.universalrejection.org}), {\bf N/A}, N/A (2002).

		%\item[](査読なし)%=============================
		%\item Kobo Abe$^3$, \underline{H. Yukawa}$^1$, 
				%``仔象は死んだ'', 
				%安部公房全集, {\bf 26}, 100-200, (2004).
	%\end{enumerate}
	%他5報
%end  学術雑誌等に発表した論文または著書 ====================
}

\subsection{学術雑誌等又は商業誌における解説・総説}
\newcommand{\学術雑誌等または商業誌における解説や総説}{%
%begin  学術雑誌等または商業誌における解説や総説===================
	
	特になし.
	
	%\begin{enumerate}
		%\item R.~Kipling, \underline{H. Yukawa},
				%``The Elephant's Child (象の鼻はなぜ長い)'', 
				%Nature, {\bf 999}, 777-779, (2003).
	%\end{enumerate}
	%他2件
%end  学術雑誌等または商業誌における解説や総説 ====================
}

\subsection{国際会議における発表}
\newcommand{\国際会議における発表}{%
%begin  国際会議における発表===================
	\vspace{10\baselineskip}
	
	%\begin{enumerate}
		%\item $\circ$ 湯川秀樹、
			%``Theory of Elephant Eggs'', 
			%原始殻物理国際会議、
			%カラチ、2006年2月

%		\item $\circ$ 湯川秀樹、Jacques-Yves Cousteau,
%			``How to search for whale eggs'',
%			国際海洋探索会議、ハワイ、2003年4月
	%\end{enumerate}
	%他1件
%end  国際会議における発表 ====================
}

\subsection{国内学会・シンポジウムにおける発表}
\newcommand{\国内学会やシンポジウムにおける発表}{%
%begin  国内学会やシンポジウムにおける発表===================
	\vspace{10\baselineskip}
	
	%\begin{enumerate}
		%\item $\circ$ 湯川秀樹、朝永振一郎、
			%「ほ乳類の真の意味」、
			%ほ乳類学会、
			%東京、2003年6月
	%\end{enumerate}
	%他3件
%end  国内学会やシンポジウムにおける発表 ====================
}

\subsection{特許}
\newcommand{\特許等}{%
%begin  特許等===================
	
	特になし.

	%\begin{enumerate}
		%\item[](公開中)
		%\item 800800号、「クジラの卵を用いた深海潜水艇」\underline{湯川秀樹}、2003年4月
%		\item[] (申請中)
%		\item 8000000号、「象の卵を用いた(ひ・み・つ)」、\underline{湯川秀樹}、2007年4月
	%\end{enumerate}		
%end  特許等 ====================
}

\subsection{その他の業績}
\newcommand{\その他の業績}{%
%begin  その他の業績===================
		%\begin{enumerate}
			%\item もうすぐもらえるで賞
		%\end{enumerate}
%end  その他の業績 ====================
}

%===========================================================
% hook9 : right before \end{document} ============

%endUserFiles
% hook7 : right before including forms ============
 % for future maintenance

% pdra_forms
%=======================================
\ifthenelse{\boolean{BudgetSummary}\OR\boolean{klTypesetPage0}}{
	%============================================================
%  Warning cover page
%============================================================

\begin{picture}(0,0)(\KLOddPictureX,\KLPictureY)
	\KLParbox{100}{700}{550}{600}{t}{
		\LARGE
		提出前に次の行を以下のようにコメントアウトし、\\
		コンパイルし直してください。\\
		\hspace{2cm}\%\textbackslash setboolean\{BudgetSummary\}\{true\}\\
		\hspace{2cm}\%\textbackslash KLTypesetPage\{..\}\\
		\hspace{2cm}\%\textbackslash KLTypesetPagesInRange\{..\}\{..\}\\
	}
	\西暦
	\KLParbox{100}{550}{500}{500}{t}{
		\begin{center}
			\LARGE 予算と研究組織のまとめ \\
			\Large \today
		\end{center}
	}

	\KLTextBox{100}{500}{550}{300}{}{
		\Large
		研究種目: \研究種目\研究種別\研究種目後半\\
		研究期間: \研究開始年度(H\研究開始元号年度) 〜 H\研究期間の最終元号年度\\
		研究課題名:「\研究課題名」\\
		研究代表者:\研究代表者氏名\\
		研究機関名:\研究機関名\\
	}
\end{picture}
\clearpage


}{}

\KLInputIfPageInRangeIsSelected{1}{2}{forms/pdra_form_04-05}
\KLInputIfPageInRangeIsSelected{3}{4}{forms/pdra_form_06-07}
\KLInputIfSelected{5}{forms/pdra_form_08}
\KLInputIfSelected{6}{forms/pdra_form_09}
\KLInputIfPageInRangeIsSelected{7}{8}{forms/pdra_form_10-11}
	
%========================================


%endFormatFile

% hook9 : right before \end{document} ============
 % for future maintenance
\end{document}
